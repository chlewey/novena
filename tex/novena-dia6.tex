\documentclass[spanish,utf8,12pt]{chlart}
\usepackage{color}
\title{Novena de Aguinaldos\\Día sexto}
\author{Fray Fernando de Jesús Larrea, Madre María Ignacia}
\date{21 de diciembre}
\definecolor{notes}{rgb}{0.3,0.26,0.2}
\definecolor{Gray}{rgb}{0.3,0.3,0.3}
\definecolor{lector}{rgb}{0,0,0}
\definecolor{responden}{rgb}{0,0.2,0.4}
\newenvironment{intro}{\begingroup
	\sffamily%
	\setlength{\leftskip}{2em}\setlength{\rightskip}{2em}\noindent
	}{\par\endgroup}
\newenvironment{summary}{\begingroup
	\footnotesize\color{notes}\sffamily\itshape
	\setlength{\leftskip}{3em}\setlength{\rightskip}{3em}\noindent
	}{\par\endgroup}
\newenvironment{lectura}{\begingroup\color{lector}}{\endgroup\par}
\newenvironment{finalnotes}{\begingroup
	\footnotesize\sffamily\color{Gray}%
	\setlength{\leftskip}{3em}\setlength{\rightskip}{3em}\noindent
	}{\par\endgroup}
\newenvironment{uno}{%{\small\sffamily\itshape\noindent(introduce)}%
	\begin{verse}\color{lector}}{\end{verse}}
\newenvironment{todos}{%{\small\sffamily\itshape\noindent(todos responden)}%
	\begin{verse}\color{responden}}{\end{verse}}
\newenvironment{gozo}{\begin{verse}\color{lector}}{\end{verse}}
\newcommand*\vena{{\color{responden}\hspace{1em}¡Ven a nuestras almas!\\\hspace{1em}¡Ven no tardes tanto!}}

\begin{document}
\maketitle

\begin{gozo}
En el nombre del padre, del hijo, y del espíritu santo.\\Amén
\end{gozo}
\section{Oración para todos los días}

\begin{lectura}
Benignísimo Dios de infinita caridad, que tanto amasteis a los hombres,
que les disteis en vuestro Hijo la mejor prenda de vuestro amor para que
hecho hombre en las entrañas de una Virgen, naciese en un pesebre para
nuestra salud y remedio; yo, en nombre de todos los mortales, os doy
infinitas gracias por tan soberano beneficio.

En retorno de él te ofrezco la pobreza, humildad y demás virtudes de
vuestro hijo humanado; suplicándoos por sus divinos méritos, por las
incomodidades con que nació y por las tiernas lágrimas que derramó en
el pesebre, que dispongáis nuestros corazones con humildad profunda,
con amor encendido, con total desprecio de todo lo terreno, para que
Jesús recién nacido tenga en ellos su cuna y more eternamente.

Amén.
\end{lectura}
\begin{finalnotes}
(Se reza tres veces el Gloria al Padre)
\end{finalnotes}

\section{Consideraciones día sexto}
\begin{lectura}
Jesús había sido concebido en Nazaret, domicilio de San José y de María,
y allí era de creerse que había de nacer, según todas las probabilidades.
Más Dios lo tenía dispuesto de otra manera y los profetas habían
anunciado que el Mesías nacería en Belén de Judá, ciudad de David.
Para que se cumpliese esa predicción, Dios se sirvió de un medio que no
parecía tener ninguna relación con este objeto, a saber:
la orden dada por el emperador Augusto de que todos los súbditos del
imperio romano se empadronasen en el lugar de donde eran originarios.
María y José como descendientes que eran de David, no estaban
dispensados de ir a Belén, y ni la situación de la Virgen Santísima ni
la necesidad en que estaba José del trabajo diario que les aseguraba la
subsistencia, pudo eximirles de este largo y penoso viaje, la estación
más rigurosa e incómoda del año.
No ignoraba Jesús en qué lugar debería nacer e inspiraba a sus padres
que se entreguen a la Providencia, y que de esta manera concurran
inconscientemente a la ejecución de sus designios.
Almas interiores observad este manejo del divino Niño, porque es el más
importante de la vida espiritual:
aprended que quien se haya entregado a Dios ya no ha de pertenecerse a
sí mismo, ni ha de querer en cada instante sino lo que Dios quiera para
él; siguiéndole ciegamente aún en las cosas exteriores, tales como el
cambio de lugar donde quiera que le plazca conducirle.
Ocasión tendréis de observar esta dependencia y esta fidelidad
inviolable en toda la vida de Jesucristo, y este es el punto sobre el
cual se han esmerado en imitarle los santos y las almas verdaderamente
interiores, renunciando absolutamente a su propia voluntad.
\end{lectura}

\section{Oración a la Santísima Virgen}
\begin{lectura}
Soberana María, que por vuestras grandes virtudes y especialmente por
vuestra humildad, merecisteis que todo un Dios os escogiese por madre
suya, os suplico que vos misma preparéis y dispongáis mi alma, y la de
todos los que en este tiempo hiciesen esta novena, para el nacimiento
espiritual de vuestro adorado Hijo.

¡Oh dulcísima Madre!
Comunicadme algo del profundo recogimiento y divina ternura con la que
le aguardasteis vos, para que nos hagáis menos indignos de verle, amarle
y adorarle por toda la eternidad.

Amén.
\end{lectura}
\begin{finalnotes}
(Se reza nueve veces el Ave María.)
\end{finalnotes}

\section{Oración a San José}
\begin{lectura}
Oh, Santísimo José!
Esposo de María y padre putativo de Jesús.
Infinitas gracias doy a Dios porque os escogió para tan altos misterios
y os adornó con todos los dones proporcionados a tan excelente grandeza.

Os ruego, por el amor que tuvisteis al Divino Niño, me abraséis en
fervorosos deseos de verle y recibirle sacramentalmente, mientras en su
divina esencia le veo y le gozo en el cielo.

Amén.
\end{lectura}
\begin{finalnotes}
(Se reza Padre nuestro, Ave María y Gloria al Padre)
\end{finalnotes}

\section{Gozos}

\begin{gozo}
Dulce Jesús Mío, mi niño adorado,\\
\vena
\end{gozo}

\begin{gozo}
¡Oh Sapiencia suma\\del Dios soberano,\\
que al nivel de un Niño\\te hayas rebajado!\\
¡Oh Divino Niño,\\ven para enseñarnos\\
la prudencia que hace\\verdaderos sabios!\\
\vena
\end{gozo}

\begin{gozo}
¡Oh, Adonaí potente\\
que a Moisés hablando,\\
de Israel al pueblo\\
dísteis los mandatos!\\
¡Ah, ven prontamente\\
para rescatarnos,\\
y que un niño débil\\
muestre fuerte brazo!\\
\vena
\end{gozo}

\begin{gozo}
¡Oh raíz sagrada\\
de Jesé, que en lo alto\\
presentas al orbe\\
tu fragante nardo!\\
¡Dulcísimo Niño,\\
que has sido llamado\\
“Lirio de los valles,\\
Bella flor del campo!”\\
\vena
\end{gozo}

\begin{gozo}
¡Llave de David\\
que abre al desterrado\\
las cerradas puertas\\
de regio palacio!\\
¡Sácanos, oh Niño,\\
con tu blanca mano,\\
de la cárcel triste\\
que labró el pecado!\\
\vena
\end{gozo}

\begin{gozo}
¡Oh, lumbre de Oriente,\\
Sol de eternos rayos,\\
que entre las tinieblas\\
tu esplendor veamos!\\
¡Niño tan precioso,\\
dicha del cristiano,\\
luzca la sonrisa\\
de tus dulces labios!\\
\vena
\end{gozo}

\begin{gozo}
¡Espejo sin mancha,\\
santo de los santos,\\
sin igual imagen del\\
Dios soberano!\\
¡Borra nuestras culpas,\\
salva al desterrado\\
y, en forma de niño,\\
da al mísero amparo!\\
\vena
\end{gozo}

\begin{gozo}
Rey de las naciones,\\
Emmanuel preclaro,\\
de Israel anhelo,\\
pastor de rebaño!\\
¡Niño que apacientas\\
con suave cayado\\
ya la oveja arisca\\
ya el cordero manso!\\
\vena
\end{gozo}

\begin{gozo}
Ábranse los cielos\\
y llueva de lo alto\\
bienhechor rocío\\
como riego santo!\\
¡Ven hermoso Niño!\\
¡Ven Dios humanado!\\
¡Luce, hermosa estrella!\\
¡Brota flor del campo!\\
\vena
\end{gozo}

\begin{gozo}
¡Ven, que ya María\\
previene sus brazos,\\
do su Niño vea\\
en tiempo cercano!\\
¡Ven que ya José\\
con anhelo sacro\\
se dispone a hacerse\\
de tu amor sagrario!\\
\vena
\end{gozo}

\begin{gozo}
¡Del débil auxilio,\\
del doliente amparo,\\
consuelo del triste,\\
luz del desterrado!\\
¡Vida de mi vida,\\
mi sueño adorado,\\
mi constante amigo,\\
mi divino hermano!\\
\vena
\end{gozo}

\begin{gozo}
¡Vén ante mis ojos\\
de ti enamorados!\\
¡Bese ya tus plantas!\\
¡Bese ya tus manos!\\
Prosternado en tierra\\
te tiendo los brazos\\
y aun más que mis frases\\
te dice mi llanto.\\
\vena
\end{gozo}

\begin{gozo}
¡Ven Salvador Nuestro\\
Por quien suspiramos!\\
\vena
\end{gozo}

\section{Oración al Niño Jesús}
\begin{lectura}
Acordaos, ¡oh dulcísimo Niño Jesús!, que dijiste a la Venerable
Margarita del Santísimo Sacramento y en persona suya a todos vuestros
devotos estas palabras tan consoladoras para nuestra pobre humanidad tan
agobiada y doliente:
“Todo lo que quieras pedir, pídelo por los méritos de mi infancia y nada
te será negado”.

Llenos de confianza en Vos, ¡Oh Jesús!, que sois la misma verdad,
venimos a exponeros toda nuestra miseria, Ayúdanos a llevar una vida
santa, para conseguir una eternidad bienaventurada.
Concedenos por los méritos infinitos de vuestra encarnación y de vuestra
infancia, la gracia de la cual necesitamos tanto.
Nos entregamos a Vos, ¡oh Niño omnipotente!
Seguros de que no quedará frustrada nuestra esperanza y de que en virtud
de vuestra divina promesa, acogeréis y despachareis favorablemente
nuestra súplica.

Amén.
\end{lectura}

\begin{gozo}
En el nombre del padre, del hijo, y del espíritu santo.\\Amén
\end{gozo}

\end{document}
