\documentclass[spanish,utf8,12pt]{chlart}
\usepackage{color}
\title{Novena de Aguinaldos\\Día noveno}
\author{Fray Fernando de Jesús Larrea, Madre María Ignacia}
\date{24 de diciembre}
\definecolor{notes}{rgb}{0.3,0.26,0.2}
\definecolor{Gray}{rgb}{0.3,0.3,0.3}
\definecolor{lector}{rgb}{0,0,0}
\definecolor{responden}{rgb}{0,0.2,0.4}
\newenvironment{intro}{\begingroup
	\sffamily%
	\setlength{\leftskip}{2em}\setlength{\rightskip}{2em}\noindent
	}{\par\endgroup}
\newenvironment{summary}{\begingroup
	\footnotesize\color{notes}\sffamily\itshape
	\setlength{\leftskip}{3em}\setlength{\rightskip}{3em}\noindent
	}{\par\endgroup}
\newenvironment{lectura}{\begingroup\color{lector}}{\endgroup\par}
\newenvironment{finalnotes}{\begingroup
	\footnotesize\sffamily\color{Gray}%
	\setlength{\leftskip}{3em}\setlength{\rightskip}{3em}\noindent
	}{\par\endgroup}
\newenvironment{uno}{%{\small\sffamily\itshape\noindent(introduce)}%
	\begin{verse}\color{lector}}{\end{verse}}
\newenvironment{todos}{%{\small\sffamily\itshape\noindent(todos responden)}%
	\begin{verse}\color{responden}}{\end{verse}}
\newenvironment{gozo}{\begin{verse}\color{lector}}{\end{verse}}
\newcommand*\vena{{\color{responden}\hspace{1em}¡Ven a nuestras almas!\\\hspace{1em}¡Ven no tardes tanto!}}

\begin{document}
\maketitle

\begin{gozo}
En el nombre del padre, del hijo, y del espíritu santo.\\Amén
\end{gozo}
\section{Oración para todos los días}

\begin{lectura}
Benignísimo Dios de infinita caridad, que tanto amasteis a los hombres,
que les disteis en vuestro Hijo la mejor prenda de vuestro amor para que
hecho hombre en las entrañas de una Virgen, naciese en un pesebre para
nuestra salud y remedio; yo, en nombre de todos los mortales, os doy
infinitas gracias por tan soberano beneficio.

En retorno de él te ofrezco la pobreza, humildad y demás virtudes de
vuestro hijo humanado; suplicándoos por sus divinos méritos, por las
incomodidades con que nació y por las tiernas lágrimas que derramó en
el pesebre, que dispongáis nuestros corazones con humildad profunda,
con amor encendido, con total desprecio de todo lo terreno, para que
Jesús recién nacido tenga en ellos su cuna y more eternamente.

Amén.
\end{lectura}
\begin{finalnotes}
(Se reza tres veces el Gloria al Padre)
\end{finalnotes}

\section{Consideraciones día noveno}
\begin{lectura}
La noche ha cerrado del todo en las campiñas de Belén.
Desechados por los hombres y viéndose sin abrigo, María y José han
salido de la inhospitalaria población, y se han refugiado en una gruta
que se encontraba al pie de la colina.
Seguía a la Reina de los Ángeles el jumento que le había servido de
cabalgadura durante el viaje y en aquella cueva hallaron un manso buey,
dejado ahí probablemente por alguno de los caminantes que había ido a
buscar hospedaje en la ciudad.

El Divino Niño, desconocido por sus criaturas va a tener que acudir a
los irracionales para que calienten con su tibio aliento la atmósfera
helada de esa noche de invierno, y le manifiesten con esto su humilde
actitud, el respeto y la adoración que le había negado Belén.
La rojiza linterna que José tenía en la mano iluminaba tenuemente ese
paupérrimo recinto, ese pesebre lleno de paja que es figura profética de
las maravillas del altar y de la íntima y prodigiosa unión eucarística
que Jesús ha de contraer con los hombres..
María está en adoración en medio de la gruta, y así van pasando
silenciosamente las horas de esa noche llena de misterios.
Pero ha llegado la media noche y de repente vemos dentro de ese pesebre
antes vacío, al Divino Niño esperado, vaticinado, deseado durante cuatro
mil años con tan inefables anhelos.
A sus pies se postra su Santísima Madre en los transporte de una
adoración de la cual nada puede dar idea.
José también se le acerca y le rinde el homenaje con que inaugura su
misterioso e imperturbable oficio de padre putativo del redentor de los
hombres.

La multitud de ángeles que descienden del cielo a contemplar esa
maravilla sin par, deja estallar su alegría y hace vibrar en los aires
las armonías de esa "Gloria in Excelsis", que es el eco de adoración que
se produce en torno al trono del Altísimo hecha perceptible por un
instante a los oídos de la pobre tierra.
Convocados por ellos, vienen en tropel los pastores de la comarca a
adorar al "recién nacido" y a prestarle sus humildes ofrendas.

Ya brilla en Oriente la misteriosa estrella de Jacob; y ya se pone en
marcha hacia Belén la caravana espléndida de los Reyes Magos, que dentro
de pocos días vendrán a depositar a los pies del Divino Niño el oro, el
incienso y la mirra, que son símbolos de la caridad, de la oración y de
la mortificación.
Oh, adorable Niño!
Nosotros también los que hemos hecho esta novena para prepararnos al día
de vuestra Navidad, queremos ofreceros nuestra pobre adoración; no la
rechacéis:
venid a nuestras almas, venid a nuestros corazones llenos de amor.

Encended en ellos la devoción a vuestra Santa Infancia, no intermitente
y sólo circunscrita al tiempo de vuestra Navidad sino siempre y en todos
los tiempos; devoción que fiel y celosamente propagada nos conduzca a la
vida eterna, librándonos del pecado y sembrando en nosotros todas las
virtudes cristianas.
\end{lectura}

\section{Oración a la Santísima Virgen}
\begin{lectura}
Soberana María, que por vuestras grandes virtudes y especialmente por
vuestra humildad, merecisteis que todo un Dios os escogiese por madre
suya, os suplico que vos misma preparéis y dispongáis mi alma, y la de
todos los que en este tiempo hiciesen esta novena, para el nacimiento
espiritual de vuestro adorado Hijo.

¡Oh dulcísima Madre!
Comunicadme algo del profundo recogimiento y divina ternura con la que
le aguardasteis vos, para que nos hagáis menos indignos de verle, amarle
y adorarle por toda la eternidad.

Amén.
\end{lectura}
\begin{finalnotes}
(Se reza nueve veces el Ave María.)
\end{finalnotes}

\section{Oración a San José}
\begin{lectura}
Oh, Santísimo José!
Esposo de María y padre putativo de Jesús.
Infinitas gracias doy a Dios porque os escogió para tan altos misterios
y os adornó con todos los dones proporcionados a tan excelente grandeza.

Os ruego, por el amor que tuvisteis al Divino Niño, me abraséis en
fervorosos deseos de verle y recibirle sacramentalmente, mientras en su
divina esencia le veo y le gozo en el cielo.

Amén.
\end{lectura}
\begin{finalnotes}
(Se reza Padre nuestro, Ave María y Gloria al Padre)
\end{finalnotes}

\section{Gozos}

\begin{gozo}
Dulce Jesús Mío, mi niño adorado,\\
\vena
\end{gozo}

\begin{gozo}
¡Oh Sapiencia suma\\del Dios soberano,\\
que al nivel de un Niño\\te hayas rebajado!\\
¡Oh Divino Niño,\\ven para enseñarnos\\
la prudencia que hace\\verdaderos sabios!\\
\vena
\end{gozo}

\begin{gozo}
¡Oh, Adonaí potente\\
que a Moisés hablando,\\
de Israel al pueblo\\
dísteis los mandatos!\\
¡Ah, ven prontamente\\
para rescatarnos,\\
y que un niño débil\\
muestre fuerte brazo!\\
\vena
\end{gozo}

\begin{gozo}
¡Oh raíz sagrada\\
de Jesé, que en lo alto\\
presentas al orbe\\
tu fragante nardo!\\
¡Dulcísimo Niño,\\
que has sido llamado\\
“Lirio de los valles,\\
Bella flor del campo!”\\
\vena
\end{gozo}

\begin{gozo}
¡Llave de David\\
que abre al desterrado\\
las cerradas puertas\\
de regio palacio!\\
¡Sácanos, oh Niño,\\
con tu blanca mano,\\
de la cárcel triste\\
que labró el pecado!\\
\vena
\end{gozo}

\begin{gozo}
¡Oh, lumbre de Oriente,\\
Sol de eternos rayos,\\
que entre las tinieblas\\
tu esplendor veamos!\\
¡Niño tan precioso,\\
dicha del cristiano,\\
luzca la sonrisa\\
de tus dulces labios!\\
\vena
\end{gozo}

\begin{gozo}
¡Espejo sin mancha,\\
santo de los santos,\\
sin igual imagen del\\
Dios soberano!\\
¡Borra nuestras culpas,\\
salva al desterrado\\
y, en forma de niño,\\
da al mísero amparo!\\
\vena
\end{gozo}

\begin{gozo}
Rey de las naciones,\\
Emmanuel preclaro,\\
de Israel anhelo,\\
pastor de rebaño!\\
¡Niño que apacientas\\
con suave cayado\\
ya la oveja arisca\\
ya el cordero manso!\\
\vena
\end{gozo}

\begin{gozo}
Ábranse los cielos\\
y llueva de lo alto\\
bienhechor rocío\\
como riego santo!\\
¡Ven hermoso Niño!\\
¡Ven Dios humanado!\\
¡Luce, hermosa estrella!\\
¡Brota flor del campo!\\
\vena
\end{gozo}

\begin{gozo}
¡Ven, que ya María\\
previene sus brazos,\\
do su Niño vea\\
en tiempo cercano!\\
¡Ven que ya José\\
con anhelo sacro\\
se dispone a hacerse\\
de tu amor sagrario!\\
\vena
\end{gozo}

\begin{gozo}
¡Del débil auxilio,\\
del doliente amparo,\\
consuelo del triste,\\
luz del desterrado!\\
¡Vida de mi vida,\\
mi sueño adorado,\\
mi constante amigo,\\
mi divino hermano!\\
\vena
\end{gozo}

\begin{gozo}
¡Vén ante mis ojos\\
de ti enamorados!\\
¡Bese ya tus plantas!\\
¡Bese ya tus manos!\\
Prosternado en tierra\\
te tiendo los brazos\\
y aun más que mis frases\\
te dice mi llanto.\\
\vena
\end{gozo}

\begin{gozo}
¡Ven Salvador Nuestro\\
Por quien suspiramos!\\
\vena
\end{gozo}

\section{Oración al Niño Jesús}
\begin{lectura}
Acordaos, ¡oh dulcísimo Niño Jesús!, que dijiste a la Venerable
Margarita del Santísimo Sacramento y en persona suya a todos vuestros
devotos estas palabras tan consoladoras para nuestra pobre humanidad tan
agobiada y doliente:
“Todo lo que quieras pedir, pídelo por los méritos de mi infancia y nada
te será negado”.

Llenos de confianza en Vos, ¡Oh Jesús!, que sois la misma verdad,
venimos a exponeros toda nuestra miseria, Ayúdanos a llevar una vida
santa, para conseguir una eternidad bienaventurada.
Concedenos por los méritos infinitos de vuestra encarnación y de vuestra
infancia, la gracia de la cual necesitamos tanto.
Nos entregamos a Vos, ¡oh Niño omnipotente!
Seguros de que no quedará frustrada nuestra esperanza y de que en virtud
de vuestra divina promesa, acogeréis y despachareis favorablemente
nuestra súplica.

Amén.
\end{lectura}

\begin{gozo}
En el nombre del padre, del hijo, y del espíritu santo.\\Amén
\end{gozo}

\end{document}
