\documentclass[spanish,utf8,12pt]{chlart}
\usepackage{color}
\title{Novena de Aguinaldos\\Día cuarto}
\author{Fray Fernando de Jesús Larrea, Madre María Ignacia}
\date{19 de diciembre}
\definecolor{notes}{rgb}{0.3,0.26,0.2}
\definecolor{Gray}{rgb}{0.3,0.3,0.3}
\definecolor{lector}{rgb}{0,0,0}
\definecolor{responden}{rgb}{0,0.2,0.4}
\newenvironment{intro}{\begingroup
	\sffamily%
	\setlength{\leftskip}{2em}\setlength{\rightskip}{2em}\noindent
	}{\par\endgroup}
\newenvironment{summary}{\begingroup
	\footnotesize\color{notes}\sffamily\itshape
	\setlength{\leftskip}{3em}\setlength{\rightskip}{3em}\noindent
	}{\par\endgroup}
\newenvironment{lectura}{\begingroup\color{lector}}{\endgroup\par}
\newenvironment{finalnotes}{\begingroup
	\footnotesize\sffamily\color{Gray}%
	\setlength{\leftskip}{3em}\setlength{\rightskip}{3em}\noindent
	}{\par\endgroup}
\newenvironment{uno}{%{\small\sffamily\itshape\noindent(introduce)}%
	\begin{verse}\color{lector}}{\end{verse}}
\newenvironment{todos}{%{\small\sffamily\itshape\noindent(todos responden)}%
	\begin{verse}\color{responden}}{\end{verse}}
\newenvironment{gozo}{\begin{verse}\color{lector}}{\end{verse}}
\newcommand*\vena{{\color{responden}\hspace{1em}¡Ven a nuestras almas!\\\hspace{1em}¡Ven no tardes tanto!}}

\begin{document}
\maketitle

\begin{gozo}
En el nombre del padre, del hijo, y del espíritu santo.\\Amén
\end{gozo}
\section{Oración para todos los días}

\begin{lectura}
Benignísimo Dios de infinita caridad, que tanto amasteis a los hombres,
que les disteis en vuestro Hijo la mejor prenda de vuestro amor para que
hecho hombre en las entrañas de una Virgen, naciese en un pesebre para
nuestra salud y remedio; yo, en nombre de todos los mortales, os doy
infinitas gracias por tan soberano beneficio.

En retorno de él te ofrezco la pobreza, humildad y demás virtudes de
vuestro hijo humanado; suplicándoos por sus divinos méritos, por las
incomodidades con que nació y por las tiernas lágrimas que derramó en
el pesebre, que dispongáis nuestros corazones con humildad profunda,
con amor encendido, con total desprecio de todo lo terreno, para que
Jesús recién nacido tenga en ellos su cuna y more eternamente.

Amén.
\end{lectura}
\begin{finalnotes}
(Se reza tres veces el Gloria al Padre)
\end{finalnotes}

\section{Consideraciones día cuarto}
\begin{lectura}
Desde el seno de su madre comenzó el Niño Jesús a poner en práctica su
entera sumisión a Dios, que continuó sin la menor interrupción durante
toda su vida.
Adoraba a su Eterno Padre, le amaba, se sometía a su voluntad; aceptaba
con resignación el estado en que se hallaba conociendo toda su
debilidad, toda su humillación, todas sus incomodidades.
¿Quién de nosotros quisiera retroceder a un estado semejante con el
pleno goce de la razón y de la reflexión?, ¿quién pudiera sostener a
sabiendas un martirio tan prolongado, tan penoso de todas maneras?
Por ahí entró el Divino Niño en su dolorosa y humilde carrera; así
empezó a anonadarse delante de su Padre, a enseñarnos lo que Dios merece
por parte de su criatura, a expiar nuestro orgullo, origen de todos
nuestros pecados y hacernos sentir toda la criminalidad y desórdenes del
orgullo.
Deseamos hacer una verdadera oración; empecemos por formarnos de ella
una exacta idea contemplando al Niño en el seno de su madre.
El divino Niño ora y ora del modo más excelente.
No habla, no medita ni se deshace en tiernos afectos.
Su mismo estado, aceptado con la intención de honrar a Dios, es su
oración y ese estado expresa altamente todo lo que Dios merece y de qué
modo quiere ser adorado de nosotros.
Unámonos a las oraciones del Niño Dios en el seno de María; unámonos al
profundo abatimiento y sea este el primer efecto de nuestro sacrificio a
Dios.
Démonos a dios no para ser algo como lo pretende continuamente nuestra
vanidad sino para ser nada, para quedar enteramente consumidos y
anonadados, para renunciar a la estimación de nosotros mismos, a todo
cuidado de nuestra grandeza aunque sea espiritual, a todo movimiento de
vanagloria.
Desaparezcamos a nuestros propios ojos y que Dios sólo sea todo para
nosotros.
\end{lectura}

\section{Oración a la Santísima Virgen}
\begin{lectura}
Soberana María, que por vuestras grandes virtudes y especialmente por
vuestra humildad, merecisteis que todo un Dios os escogiese por madre
suya, os suplico que vos misma preparéis y dispongáis mi alma, y la de
todos los que en este tiempo hiciesen esta novena, para el nacimiento
espiritual de vuestro adorado Hijo.

¡Oh dulcísima Madre!
Comunicadme algo del profundo recogimiento y divina ternura con la que
le aguardasteis vos, para que nos hagáis menos indignos de verle, amarle
y adorarle por toda la eternidad.

Amén.
\end{lectura}
\begin{finalnotes}
(Se reza nueve veces el Ave María.)
\end{finalnotes}

\section{Oración a San José}
\begin{lectura}
Oh, Santísimo José!
Esposo de María y padre putativo de Jesús.
Infinitas gracias doy a Dios porque os escogió para tan altos misterios
y os adornó con todos los dones proporcionados a tan excelente grandeza.

Os ruego, por el amor que tuvisteis al Divino Niño, me abraséis en
fervorosos deseos de verle y recibirle sacramentalmente, mientras en su
divina esencia le veo y le gozo en el cielo.

Amén.
\end{lectura}
\begin{finalnotes}
(Se reza Padre nuestro, Ave María y Gloria al Padre)
\end{finalnotes}

\section{Gozos}

\begin{gozo}
Dulce Jesús Mío, mi niño adorado,\\
\vena
\end{gozo}

\begin{gozo}
¡Oh Sapiencia suma\\del Dios soberano,\\
que al nivel de un Niño\\te hayas rebajado!\\
¡Oh Divino Niño,\\ven para enseñarnos\\
la prudencia que hace\\verdaderos sabios!\\
\vena
\end{gozo}

\begin{gozo}
¡Oh, Adonaí potente\\
que a Moisés hablando,\\
de Israel al pueblo\\
dísteis los mandatos!\\
¡Ah, ven prontamente\\
para rescatarnos,\\
y que un niño débil\\
muestre fuerte brazo!\\
\vena
\end{gozo}

\begin{gozo}
¡Oh raíz sagrada\\
de Jesé, que en lo alto\\
presentas al orbe\\
tu fragante nardo!\\
¡Dulcísimo Niño,\\
que has sido llamado\\
“Lirio de los valles,\\
Bella flor del campo!”\\
\vena
\end{gozo}

\begin{gozo}
¡Llave de David\\
que abre al desterrado\\
las cerradas puertas\\
de regio palacio!\\
¡Sácanos, oh Niño,\\
con tu blanca mano,\\
de la cárcel triste\\
que labró el pecado!\\
\vena
\end{gozo}

\begin{gozo}
¡Oh, lumbre de Oriente,\\
Sol de eternos rayos,\\
que entre las tinieblas\\
tu esplendor veamos!\\
¡Niño tan precioso,\\
dicha del cristiano,\\
luzca la sonrisa\\
de tus dulces labios!\\
\vena
\end{gozo}

\begin{gozo}
¡Espejo sin mancha,\\
santo de los santos,\\
sin igual imagen del\\
Dios soberano!\\
¡Borra nuestras culpas,\\
salva al desterrado\\
y, en forma de niño,\\
da al mísero amparo!\\
\vena
\end{gozo}

\begin{gozo}
Rey de las naciones,\\
Emmanuel preclaro,\\
de Israel anhelo,\\
pastor de rebaño!\\
¡Niño que apacientas\\
con suave cayado\\
ya la oveja arisca\\
ya el cordero manso!\\
\vena
\end{gozo}

\begin{gozo}
Ábranse los cielos\\
y llueva de lo alto\\
bienhechor rocío\\
como riego santo!\\
¡Ven hermoso Niño!\\
¡Ven Dios humanado!\\
¡Luce, hermosa estrella!\\
¡Brota flor del campo!\\
\vena
\end{gozo}

\begin{gozo}
¡Ven, que ya María\\
previene sus brazos,\\
do su Niño vea\\
en tiempo cercano!\\
¡Ven que ya José\\
con anhelo sacro\\
se dispone a hacerse\\
de tu amor sagrario!\\
\vena
\end{gozo}

\begin{gozo}
¡Del débil auxilio,\\
del doliente amparo,\\
consuelo del triste,\\
luz del desterrado!\\
¡Vida de mi vida,\\
mi sueño adorado,\\
mi constante amigo,\\
mi divino hermano!\\
\vena
\end{gozo}

\begin{gozo}
¡Vén ante mis ojos\\
de ti enamorados!\\
¡Bese ya tus plantas!\\
¡Bese ya tus manos!\\
Prosternado en tierra\\
te tiendo los brazos\\
y aun más que mis frases\\
te dice mi llanto.\\
\vena
\end{gozo}

\begin{gozo}
¡Ven Salvador Nuestro\\
Por quien suspiramos!\\
\vena
\end{gozo}

\section{Oración al Niño Jesús}
\begin{lectura}
Acordaos, ¡oh dulcísimo Niño Jesús!, que dijiste a la Venerable
Margarita del Santísimo Sacramento y en persona suya a todos vuestros
devotos estas palabras tan consoladoras para nuestra pobre humanidad tan
agobiada y doliente:
“Todo lo que quieras pedir, pídelo por los méritos de mi infancia y nada
te será negado”.

Llenos de confianza en Vos, ¡Oh Jesús!, que sois la misma verdad,
venimos a exponeros toda nuestra miseria, Ayúdanos a llevar una vida
santa, para conseguir una eternidad bienaventurada.
Concedenos por los méritos infinitos de vuestra encarnación y de vuestra
infancia, la gracia de la cual necesitamos tanto.
Nos entregamos a Vos, ¡oh Niño omnipotente!
Seguros de que no quedará frustrada nuestra esperanza y de que en virtud
de vuestra divina promesa, acogeréis y despachareis favorablemente
nuestra súplica.

Amén.
\end{lectura}

\begin{gozo}
En el nombre del padre, del hijo, y del espíritu santo.\\Amén
\end{gozo}

\end{document}
