\documentclass[spanish,utf8,twocolumn]{chlart}
\usepackage{color}
\title{Novena de Aguinaldos}
\author{Fray Fernando de Jesús Larrea, Madre María Ignacia}
\date{December 2014}
\definecolor{notes}{rgb}{0.3,0.26,0.2}
\definecolor{Gray}{rgb}{0.3,0.3,0.3}
\definecolor{lector}{rgb}{0,0,0}
\definecolor{responden}{rgb}{0,0.2,0.4}
\newenvironment{intro}{\begingroup
	\sffamily%
	\setlength{\leftskip}{2em}\setlength{\rightskip}{2em}\noindent
	}{\par\endgroup}
\newenvironment{summary}{\begingroup
	\small\sffamily\itshape%
	\setlength{\leftskip}{3em}\setlength{\rightskip}{3em}\noindent
	}{\par\endgroup}
\newenvironment{lectura}{\begingroup\color{lector}}{\endgroup\par}
\newenvironment{finalnotes}{\begingroup
	\footnotesize\sffamily\color{Gray}%
	\setlength{\leftskip}{3em}\setlength{\rightskip}{3em}\noindent
	}{\par\endgroup}
\newenvironment{uno}{%{\small\sffamily\itshape\noindent(introduce)}%
	\begin{verse}\color{lector}}{\end{verse}}
\newenvironment{todos}{%{\small\sffamily\itshape\noindent(todos responden)}%
	\begin{verse}\color{responden}}{\end{verse}}
\newenvironment{gozo}{\begin{verse}\color{lector}}{\end{verse}}
\newcommand*\vena{{\color{responden}\hspace{1em}¡Ven a nuestras almas!\\\hspace{1em}¡Ven no tardes tanto!}}
\begin{document}
\color{notes}
\maketitle

\begin{intro}
La Novena de Aguinaldos es una costumbre católica arraigada en Colombia,
Venezuela y Ecuador que está relacionada con la festividad de Navidad.
Se trata de una oración rezada durante nueve días (novena) en la época
de los aguinaldos (previa a la Navidad).

La novena fue creada por Fray Fernando de Jesús Larrea nacido en Quito
en 1700 quien después de su ordenación en 1725 fue predicador en Ecuador
y Colombia.
Fray Fernando la escribió por petición de la fundadora del
Colegio de La Enseñanza en Bogotá doña Clemencia de Jesús Caycedo Vélez.

Muchos años después una religiosa de La Enseñanza, la madre María
Ignacia, cuyo nombre era realmente Bertilda Samper Acosta, la modificó
y agregó los gozos.

En la novena de Aguinaldos se reza durante 9 días desde el 16 hasta el
24 de diciembre, rememorando los meses previos al nacimiento de Jesús y
terminando con su llegada en el pesebre de Belén.

Cada día se reza un conjunto de oraciones que son:
\begin{enumerate}
  \item  Oración Para Todos Los Días
  \item  Consideraciones diarias
  \item  Oración a la Santísima Virgen
  \item  Oración a San José
  \item  Gozos
  \item  Oración al Niño Jesús
\end{enumerate}

Como guía de lectura, el texto en \textrm{\color{lector}negro} será
leído en voz alta por el respectivo lector, y el texto en
\textrm{\color{responden}azul} será respondido por todos los presentes.
\end{intro}

\begin{gozo}
En el nombre del padre, del hijo, y del espíritu santo.\\Amén
\end{gozo}
\section{Oración para todos los días}
\begin{summary}
Se reza todos los días al inicio de la novena.
\end{summary}
\begin{lectura}
Benignísimo Dios de infinita caridad, que tanto amasteis a los hombres,
que les disteis en vuestro Hijo la mejor prenda de vuestro amor para que
hecho hombre en las entrañas de una Virgen, naciese en un pesebre para
nuestra salud y remedio; yo, en nombre de todos los mortales, os doy
infinitas gracias por tan soberano beneficio.

En retorno de él te ofrezco la pobreza, humildad y demás virtudes de
vuestro hijo humanado; suplicándoos por sus divinos méritos, por las
incomodidades con que nació y por las tiernas lágrimas que derramó en
el pesebre, que dispongáis nuestros corazones con humildad profunda,
con amor encendido, con total desprecio de todo lo terreno, para que
Jesús recién nacido tenga en ellos su cuna y more eternamente.

Amén.
\end{lectura}
\begin{finalnotes}
(Se reza tres veces el Gloria al Padre)
\end{finalnotes}

\section{Consideraciones diarias}
\begin{summary}
Se lee la consideración del día respectivo justo después de la
oración para todos los días.
\end{summary}

\subsection*{Día primero}
\begin{summary}
Se lee el 16 de diciembre.
\end{summary}
\begin{lectura}
En el principio de los tiempos el Verbo reposaba en el seno de su Padre
en lo más alto de los cielos:
allí era la causa, a la par que el modelo de toda creación.
En esas profundidades de una incalculable eternidad permanecía el Niño
de Belén.
Allí es donde debemos datar la genealogía del Eterno que no tiene
antepasados, y contemplar la vida de complacencia infinita que allí
llevaba.

La vida del Verbo Eterno en el seno de su Padre era una vida maravillosa
y sin embargo, misterio sublime, busca otra morada en una mansión
creada.
No era porque en su mansión eterna faltase algo a su infinita felicidad
sino porque su misericordia infinita anhelaba la redención y la
salvación del género humano, que sin Él no podría verificarse.

El pecado de Adán había ofendido a un Dios y esa ofensa infinita no
podría ser condonada sino por los méritos del mismo Dios.
La raza de Adán había desobedecido y merecido un castigo eterno; era
pues, necesario para salvarla y satisfacer su culpa que Dios, sin dejar
el cielo, tomase la forma del hombre sobre la tierra y con la obediencia
a los designios de su Padre, expiase aquella desobediencia, ingratitud y
rebeldía.

Era necesario en las miras de su amor que tomase la forma, las
debilidades e ignorancia sistemática del hombre, que creciese para darle
crecimiento espiritual; que sufriese, para morir a sus pasiones y a su
orgullo y por eso el Verbo Eterno ardiendo en deseos de salvar al hombre
resolvió hacerse hombre también y así redimir al culpable.
\end{lectura}
\subsection*{Día segundo}
\begin{summary}
Se lee el 17 de diciembre.
\end{summary}
\begin{lectura}
El verbo eterno se halla a punto de tomar su naturales creada en la
santa casa de Nazaret, en donde moraban María y José.
Cuando la sombra del decreto divino vino a deslizarse sobre ella, María
estaba sola y engolfada en la oración.
Pasaba las silenciosas horas de la noche en la unión más estrecha con
Dios; y mientras oraba, el Verbo tomó posesión de su morada creada.
Sin embargo, no llegó inopinadamente:
antes de presentarse envió a un mensajero, que fue Arcángel San Gabriel
para pedir a María de parte de Dios su consentimiento para la encarnación.
El creador no quiso efectuar ese gran misterio sin la aquiescencia de su
criatura.

Aquel momento fue muy solemne:
era potestativo en María rehusar...
Con qué adorables delicias, con qué inefable complacencia aguardaría la
Santísima Trinidad a que María abriese los labios y pronunciase el "sí"
que debió ser suave melodía para sus oídos, y con el cual se conformaba
su profunda humildad a la omnipotente voluntad divina.
La Virgen Inmaculada ha dado su asentimiento.
El arcángel ha desaparecido.
Dios se ha revestido de una naturaleza creada; la voluntad eterna está
cumplida y la creación completa.
En las regiones del mundo angélico estalla el júbilo inmenso, pero la
Virgen María ni le oía ni le hubiese prestado atención a él.
Tenía inclinada la cabeza y su alma estaba sumida en el silencio que se
asemejaba al de Dios.
El Verbo se había hecho carne, y aunque todavía invisible para el mundo,
habitaba ya entre los hombres que su inmenso amor había venido a
rescatar.
No era ya sólo el Verbo eterno; era el Niño Jesús revestido de la
apariencia humana, y justificando ya el elogio que de Él han hecho todas
las generaciones en llamarle el más hermoso de los hijos de los hombres.
\end{lectura}
\subsection*{Día tercero}
\begin{summary}
Se lee el 18 de diciembre.
\end{summary}
\begin{lectura}
Así había comenzado su vida encarnada el Niño.
Consideremos el alma gloriosa y el santo cuerpo que había tomado,
adorándolos profundamente.
Admirando en el primer lugar el alma de ese divino Niño, consideremos en
ella la plenitud de su gracia santificadora; la de su ciencia beatífica,
por la cual desde el primer momento de su vida vio la divina esencia más
claramente que todos los ángeles y leyó lo pasado lo porvenir con todos
sus arcanos conocimientos.
No supo nunca por adquisición voluntaria nada que no supiese por
infusión desde el primer momento de su ser; pero él adoptó todas las
enfermedades de nuestra naturaleza a que dignamente podía someterse, aún
cuando no fuesen necesarias para grande obra que debía cumplir.
Pidámosle que sus divinas facultades suplan la debilidad de las nuestras
y les den nueva energía; que su memoria nos enseñe a recordar sus
beneficios, su entendimiento a pensar en Él, su voluntad a no hacer sino
lo que Él quiere y en servicio suyo.
Del alma del Niño Jesús pasemos ahora a su cuerpo.
Que era un mundo de maravillas, una obra maestra de la mano de Dios.
No era, como el nuestro, una traba para el alma:
era por el contrario, un nuevo elemento de santidad.
Quiso que fuese pequeño y débil como el de todos los niños, y sujeto a
todas las incomodidades de la infancia, para asemejarse más a nosotros y
participar de nuestras humillaciones.
El Espíritu Santo formó ese cuerpecillo divino con tal delicadeza y tal
capacidad de sentir, que pudiese sufrir hasta el exceso para cumplir la
grande obra de nuestra redención.
La belleza de ese cuerpo del divino Niño fue superior a cuanto se ha
imaginado jamás; la divina sangre que por sus venas empezó a circular
desde el momento de la encarnación es la que lava todas las manchas del
mundo culpable.
Pidámosle que lave las nuestras en el sacramento de la penitencia, para
que el día de su Navidad nos encuentre purificados, perdonados y
dispuestos a recibirle con amor y provecho espiritual.
\end{lectura}
\subsection*{Día cuarto}
\begin{summary}
Se lee el 19 de diciembre.
\end{summary}
\begin{lectura}
Desde el seno de su madre comenzó el Niño Jesús a poner en práctica su
entera sumisión a Dios, que continuó sin la menor interrupción durante
toda su vida.
Adoraba a su Eterno Padre, le amaba, se sometía a su voluntad; aceptaba
con resignación el estado en que se hallaba conociendo toda su
debilidad, toda su humillación, todas sus incomodidades.
¿Quién de nosotros quisiera retroceder a un estado semejante con el
pleno goce de la razón y de la reflexión?, ¿quién pudiera sostener a
sabiendas un martirio tan prolongado, tan penoso de todas maneras?
Por ahí entró el Divino Niño en su dolorosa y humilde carrera; así
empezó a anonadarse delante de su Padre, a enseñarnos lo que Dios merece
por parte de su criatura, a expiar nuestro orgullo, origen de todos
nuestros pecados y hacernos sentir toda la criminalidad y desórdenes del
orgullo.
Deseamos hacer una verdadera oración; empecemos por formarnos de ella
una exacta idea contemplando al Niño en el seno de su madre.
El divino Niño ora y ora del modo más excelente.
No habla, no medita ni se deshace en tiernos afectos.
Su mismo estado, aceptado con la intención de honrar a Dios, es su
oración y ese estado expresa altamente todo lo que Dios merece y de qué
modo quiere ser adorado de nosotros.
Unámonos a las oraciones del Niño Dios en el seno de María; unámonos al
profundo abatimiento y sea este el primer efecto de nuestro sacrificio a
Dios.
Démonos a dios no para ser algo como lo pretende continuamente nuestra
vanidad sino para ser nada, para quedar enteramente consumidos y
anonadados, para renunciar a la estimación de nosotros mismos, a todo
cuidado de nuestra grandeza aunque sea espiritual, a todo movimiento de
vanagloria.
Desaparezcamos a nuestros propios ojos y que Dios sólo sea todo para
nosotros.
\end{lectura}
\subsection*{Día quinto}
\begin{summary}
Se lee el 20 de diciembre.
\end{summary}
\begin{lectura}
Ya hemos visto la vida que llevaba el Niño Jesús en el seno de su
purísima Madre; veamos hoy la vida que llevaba también María durante el
mismo espacio de tiempo.
Necesidad hoy de que nos detengamos en ella si queremos comprender, en
cuanto es posible a nuestra limitada capacidad, los sublimes misterios
de la encarnación y el modo como hemos de corresponder a ellos.
María no cesaba de aspirar por el momento en que gozaría de esa visión
beatífica terrestre:
la faz de Dios encarnado.
Estaba a punto de ver aquella faz humana que debía iluminar el cielo
durante toda la eternidad.
Iba a leer el amor filial en aquellos mismos ojos cuyos rayos deberían
esparcir para siempre la felicidad en millones de elegidos.
Iba a ver aquel rostro todos los días, a todas horas, cada instante,
durante muchos años.
Iba a verle en la ignorancia aparente de la infancia, en los encantos
particulares de la juventud y en la serenidad reflexiva de la edad
madura...
Haría todo lo que quisiese de aquella faz divina; podría estrecharla
contra la suya con toda la libertad del amor materno; cubrir de besos
los labios que deberían pronunciar la sentencia a todos los hombres;
contemplarla a su gusto durante su sueño o despierto, hasta que la
hubiese aprendido de memoria...
¿Cuán ardientemente deseaba ese día!
Tal era la vida de expectativa de María...
era inaudita en sí misma, más no por eso dejaba de ser el tipo magnífico
de toda vida cristiana, no nos contentemos con admirar a Jesús
residiendo en María, sino pensemos que en nosotros también reside por
esencia, potencia y presencia.
Sí, Jesús nace continuamente en nosotros y de nosotros, por las buenas
obras que nos hace capaces de cumplir, y por nuestra cooperación a la
gracia; por la manera que el alma del que se halla en gracia es un seno
perpetuo de María, un Belén interior sin fin.
Después de la comunión Jesús habita en nosotros, durante algunos
instantes, real y sustancialmente como Dios y como hombre, porque el
mismo niño que estaba en María está también en el Santísimo Sacramento.
¿Qué es todo esto sino una participación de la vida de María durante
esos maravillosos meses, y una expectativa llena de delicias como la
suya?
\end{lectura}
\subsection*{Día sexto}
\begin{summary}
Se lee el 21 de diciembre.
\end{summary}
\begin{lectura}
Jesús había sido concebido en Nazaret, domicilio de San José y de María,
y allí era de creerse que había de nacer, según todas las probabilidades.
Más Dios lo tenía dispuesto de otra manera y los profetas habían
anunciado que el Mesías nacería en Belén de Judá, ciudad de David.
Para que se cumpliese esa predicción, Dios se sirvió de un medio que no
parecía tener ninguna relación con este objeto, a saber:
la orden dada por el emperador Augusto de que todos los súbditos del
imperio romano se empadronasen en el lugar de donde eran originarios.
María y José como descendientes que eran de David, no estaban
dispensados de ir a Belén, y ni la situación de la Virgen Santísima ni
la necesidad en que estaba José del trabajo diario que les aseguraba la
subsistencia, pudo eximirles de este largo y penoso viaje, la estación
más rigurosa e incómoda del año.
No ignoraba Jesús en qué lugar debería nacer e inspiraba a sus padres
que se entreguen a la Providencia, y que de esta manera concurran
inconscientemente a la ejecución de sus designios.
Almas interiores observad este manejo del divino Niño, porque es el más
importante de la vida espiritual:
aprended que quien se haya entregado a Dios ya no ha de pertenecerse a
sí mismo, ni ha de querer en cada instante sino lo que Dios quiera para
él; siguiéndole ciegamente aún en las cosas exteriores, tales como el
cambio de lugar donde quiera que le plazca conducirle.
Ocasión tendréis de observar esta dependencia y esta fidelidad
inviolable en toda la vida de Jesucristo, y este es el punto sobre el
cual se han esmerado en imitarle los santos y las almas verdaderamente
interiores, renunciando absolutamente a su propia voluntad.
\end{lectura}
\subsection*{Día séptimo}
\begin{summary}
Se lee el 22 de diciembre.
\end{summary}
\begin{lectura}
Representémonos el viaje de María y José hacia Belén, llevando consigo
aún no nacido, al creador del universo, hecho hombre.
Contemplemos la humildad y la obediencia de ese Divino Niño, que aunque
de raza judía y habiendo amado durante siglos a su pueblo con una
predilección inexplicable obedece así a un príncipe extranjero que forma
el censo de población de su provincia, como si hubiese para él en esa
circunstancia algo que le halagase, y quisiera apresurarse a aprovechar
la ocasión de hacerse empadronar oficial y auténticamente como súbdito
en el momento en que venía al mundo.

El anhelo de José, la expectativa de María son cosas que no puede
expresar el lenguaje humano.
El Padre Eterno se halla, si nos es lícito emplear esta expresión,
adorablemente impaciente por dar a su hijo único al mundo y verle ocupar
su puesto entre las criaturas visibles.

El Espíritu Santo arde en deseos de presentar a la luz del día esa santa
humanidad, que El mismo ha formado con divino esmero.
\end{lectura}
\subsection*{Día octavo}
\begin{summary}
Se lee el 23 de diciembre.
\end{summary}
\begin{lectura}
Llegan a Belén José y María buscando hospedaje en los mesones, pero no
encuentran, ya por hallarse todos ocupados, ya porque se les deshace a
causa de su pobreza.
Empero, nada puede turbar la paz interior de los que están fijos en
Dios.

Si José experimentaba tristeza cuando era rechazado de casa en casa,
porque pensaba en María y en el Niño, sonreíase también con santa
tranquilidad cuando fijaba la mirada en su casta esposa.
El ruido de cada puerta que se cerraba ante ellos era una dulce melodía
para sus oídos.

Eso era lo que había venido a buscar.
El deseo de esas humillaciones era lo que había contribuido a hacerle
tomar la forma humana.
Oh!
Divino Niño de Belén!
Estos días que tantos han pasado en fiestas y diversiones o descansando
muellemente en cómodas y ricas mansiones, ha sido para vuestros padres
un día de fatiga y vejaciones de toda clase.
¡Ay!
el espíritu de Belén es el de un mundo que ha olvidado a Dios.

¡Cuántas veces no ha sido también el nuestro!
Pónese el sol el 24 de diciembre detrás de los tejados de Belén y sus
últimos rayos doran la cima de las rocas escarpadas que lo rodean.
Hombres groseros, codean rudamente al Señor en las calles de aquella
aldea oriental y cierran sus puertas al ver a a su Madre.

La bóveda de los cielos aparece purpurina por encima de aquellas colinas
frecuentadas por los pastores.
Las estrellas van apareciendo unas tras otras.
Algunas horas más y aparecerá el Verbo Eterno.
\end{lectura}
\subsection*{Día noveno}
\begin{summary}
Se lee el 24 de diciembre.
\end{summary}
\begin{lectura}
La noche ha cerrado del todo en las campiñas de Belén.
Desechados por los hombres y viéndose sin abrigo, María y José han
salido de la inhospitalaria población, y se han refugiado en una gruta
que se encontraba al pie de la colina.
Seguía a la Reina de los Ángeles el jumento que le había servido de
cabalgadura durante el viaje y en aquella cueva hallaron un manso buey,
dejado ahí probablemente por alguno de los caminantes que había ido a
buscar hospedaje en la ciudad.

El Divino Niño, desconocido por sus criaturas va a tener que acudir a
los irracionales para que calienten con su tibio aliento la atmósfera
helada de esa noche de invierno, y le manifiesten con esto su humilde
actitud, el respeto y la adoración que le había negado Belén.
La rojiza linterna que José tenía en la mano iluminaba tenuemente ese
paupérrimo recinto, ese pesebre lleno de paja que es figura profética de
las maravillas del altar y de la íntima y prodigiosa unión eucarística
que Jesús ha de contraer con los hombres..
María está en adoración en medio de la gruta, y así van pasando
silenciosamente las horas de esa noche llena de misterios.
Pero ha llegado la media noche y de repente vemos dentro de ese pesebre
antes vacío, al Divino Niño esperado, vaticinado, deseado durante cuatro
mil años con tan inefables anhelos.
A sus pies se postra su Santísima Madre en los transporte de una
adoración de la cual nada puede dar idea.
José también se le acerca y le rinde el homenaje con que inaugura su
misterioso e imperturbable oficio de padre putativo del redentor de los
hombres.

La multitud de ángeles que descienden del cielo a contemplar esa
maravilla sin par, deja estallar su alegría y hace vibrar en los aires
las armonías de esa "Gloria in Excelsis", que es el eco de adoración que
se produce en torno al trono del Altísimo hecha perceptible por un
instante a los oídos de la pobre tierra.
Convocados por ellos, vienen en tropel los pastores de la comarca a
adorar al "recién nacido" y a prestarle sus humildes ofrendas.

Ya brilla en Oriente la misteriosa estrella de Jacob; y ya se pone en
marcha hacia Belén la caravana espléndida de los Reyes Magos, que dentro
de pocos días vendrán a depositar a los pies del Divino Niño el oro, el
incienso y la mirra, que son símbolos de la caridad, de la oración y de
la mortificación.
Oh, adorable Niño!
Nosotros también los que hemos hecho esta novena para prepararnos al día
de vuestra Navidad, queremos ofreceros nuestra pobre adoración; no la
rechacéis:
venid a nuestras almas, venid a nuestros corazones llenos de amor.

Encended en ellos la devoción a vuestra Santa Infancia, no intermitente
y sólo circunscrita al tiempo de vuestra Navidad sino siempre y en todos
los tiempos; devoción que fiel y celosamente propagada nos conduzca a la
vida eterna, librándonos del pecado y sembrando en nosotros todas las
virtudes cristianas.
\end{lectura}
\newpage 
\section{Oración a la Santísima Virgen}
\begin{summary}
Se reza después de la consideración del día.
\end{summary}
\begin{lectura}
Soberana María, que por vuestras grandes virtudes y especialmente por
vuestra humildad, merecisteis que todo un Dios os escogiese por madre
suya, os suplico que vos misma preparéis y dispongáis mi alma, y la de
todos los que en este tiempo hiciesen esta novena, para el nacimiento
espiritual de vuestro adorado Hijo.

¡Oh dulcísima Madre!
Comunicadme algo del profundo recogimiento y divina ternura con la que
le aguardasteis vos, para que nos hagáis menos indignos de verle, amarle
y adorarle por toda la eternidad.

Amén.
\end{lectura}
\begin{finalnotes}
(Se reza nueve veces el Ave María.)
\end{finalnotes}

\section{Oración a San José}
\begin{summary}
Se reza después de la oración a la santísima virgen.
\end{summary}
\begin{lectura}
Oh, Santísimo José!
Esposo de María y padre putativo de Jesús.
Infinitas gracias doy a Dios porque os escogió para tan altos misterios
y os adornó con todos los dones proporcionados a tan excelente grandeza.

Os ruego, por el amor que tuvisteis al Divino Niño, me abraséis en
fervorosos deseos de verle y recibirle sacramentalmente, mientras en su
divina esencia le veo y le gozo en el cielo.

Amén.
\end{lectura}
\begin{finalnotes}
(Se reza Padre nuestro, Ave María y Gloria al Padre)
\end{finalnotes}

\section{Gozos}
\begin{summary}
Se cantan o se rezan después de la oración a San José.

Al finalizar cada gozo, todos cantan o rezan el estribillo \\\vena\\o cualquiera de sus variantes.
\end{summary}

\newpage Coro
\begin{gozo}
Dulce Jesús Mío, mi niño adorado,\\
\vena
\end{gozo}

1
\begin{gozo}
¡Oh Sapiencia suma\\del Dios soberano,\\
que al nivel de un Niño\\te hayas rebajado!\\
¡Oh Divino Niño,\\ven para enseñarnos\\
la prudencia que hace\\verdaderos sabios!\\
\vena
\end{gozo}

2
\begin{gozo}
¡Oh, Adonaí potente\\
que a Moisés hablando,\\
de Israel al pueblo\\
dísteis los mandatos!\\
¡Ah, ven prontamente\\
para rescatarnos,\\
y que un niño débil\\
muestre fuerte brazo!\\
\vena
\end{gozo}

3
\begin{gozo}
¡Oh raíz sagrada\\
de Jesé, que en lo alto\\
presentas al orbe\\
tu fragante nardo!\\
¡Dulcísimo Niño,\\
que has sido llamado\\
“Lirio de los valles,\\
Bella flor del campo!”\\
\vena
\end{gozo}

\newpage 4
\begin{gozo}
¡Llave de David\\
que abre al desterrado\\
las cerradas puertas\\
de regio palacio!\\
¡Sácanos, oh Niño,\\
con tu blanca mano,\\
de la cárcel triste\\
que labró el pecado!\\
\vena
\end{gozo}

5
\begin{gozo}
¡Oh, lumbre de Oriente,\\
Sol de eternos rayos,\\
que entre las tinieblas\\
tu esplendor veamos!\\
¡Niño tan precioso,\\
dicha del cristiano,\\
luzca la sonrisa\\
de tus dulces labios!\\
\vena
\end{gozo}

6
\begin{gozo}
¡Espejo sin mancha,\\
santo de los santos,\\
sin igual imagen del\\
Dios soberano!\\
¡Borra nuestras culpas,\\
salva al desterrado\\
y, en forma de niño,\\
da al mísero amparo!\\
\vena
\end{gozo}

\newpage 7
\begin{gozo}
Rey de las naciones,\\
Emmanuel preclaro,\\
de Israel anhelo,\\
pastor de rebaño!\\
¡Niño que apacientas\\
con suave cayado\\
ya la oveja arisca\\
ya el cordero manso!\\
\vena
\end{gozo}

8
\begin{gozo}
Ábranse los cielos\\
y llueva de lo alto\\
bienhechor rocío\\
como riego santo!\\
¡Ven hermoso Niño!\\
¡Ven Dios humanado!\\
¡Luce, hermosa estrella!\\
¡Brota flor del campo!\\
\vena
\end{gozo}

9
\begin{gozo}
¡Ven, que ya María\\
previene sus brazos,\\
do su Niño vea\\
en tiempo cercano!\\
¡Ven que ya José\\
con anhelo sacro\\
se dispone a hacerse\\
de tu amor sagrario!\\
\vena
\end{gozo}

\newpage 10
\begin{gozo}
¡Del débil auxilio,\\
del doliente amparo,\\
consuelo del triste,\\
luz del desterrado!\\
¡Vida de mi vida,\\
mi sueño adorado,\\
mi constante amigo,\\
mi divino hermano!\\
\vena
\end{gozo}

11
\begin{gozo}
¡Vén ante mis ojos\\
de ti enamorados!\\
¡Bese ya tus plantas!\\
¡Bese ya tus manos!\\
Prosternado en tierra\\
te tiendo los brazos\\
y aun más que mis frases\\
te dice mi llanto.\\
\vena
\end{gozo}

12
\begin{gozo}
¡Ven Salvador Nuestro\\
Por quien suspiramos!\\
\vena
\end{gozo}
\newpage 
\section{Oración al Niño Jesús}
\begin{summary}
Se reza como oración final, después de los gozos.
\end{summary}
\begin{lectura}
Acordaos, ¡oh dulcísimo Niño Jesús!, que dijiste a la Venerable
Margarita del Santísimo Sacramento y en persona suya a todos vuestros
devotos estas palabras tan consoladoras para nuestra pobre humanidad tan
agobiada y doliente:
“Todo lo que quieras pedir, pídelo por los méritos de mi infancia y nada
te será negado”.

Llenos de confianza en Vos, ¡Oh Jesús!, que sois la misma verdad,
venimos a exponeros toda nuestra miseria, Ayúdanos a llevar una vida
santa, para conseguir una eternidad bienaventurada.
Concedenos por los méritos infinitos de vuestra encarnación y de vuestra
infancia, la gracia de la cual necesitamos tanto.
Nos entregamos a Vos, ¡oh Niño omnipotente!
Seguros de que no quedará frustrada nuestra esperanza y de que en virtud
de vuestra divina promesa, acogeréis y despachareis favorablemente
nuestra súplica.

Amén.
\end{lectura}
\newpage
\begin{gozo}
En el nombre del padre, del hijo, y del espíritu santo.\\Amén
\end{gozo}

\section*{Oraciones}
\subsection*{Gloria al Padre}

\begin{uno}
Gloria al Padre y al Hijo y al Espíritu Santo
\end{uno}

\begin{todos}
Como era en un principio,\\ahora y siempre,\\por los siglos de los siglos.\\
Amén.
\end{todos}

\subsection*{Ave María}

\begin{uno}
¡Dios te salve María,\\llena eres de gracia!\\
El Señor está contigo.\\
Bendita tú eres entre todas las mujeres,\\
y bendito es el fruto de tu vientre, Jesús.
\end{uno}

\begin{todos}
¡Santa María, madre de Dios,\\
ruega por nosotros los pecadores,\\
ahora y en la hora de nuestra muerte!\\
Amén.
\end{todos}

\subsection*{Padre nuestro}

\begin{uno}
Padre nuestro que estás en el Cielo,\\
santificado sea Tu nombre.\\
Venga a nosotros Tu reino.\\
Hágase Tu voluntad\\
en la tierra como en el Cielo.
\end{uno}

\begin{todos}
Danos hoy nuestro pan de cada día.\\
Perdona nuestras ofensas,\\
como también nosotros perdonamos a los que nos ofenden.\\
No nos dejes caer en tentación\\
y líbranos del mal,\\
Amén.
\end{todos}

\end{document}

cuota 18.480
seguro 47.552
TOTAL 66.032
132.064

2809028


Carlos Alberto Gaitán
312 4563076
